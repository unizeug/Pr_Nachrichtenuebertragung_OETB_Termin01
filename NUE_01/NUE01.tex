% \newcommand{\prototitle}{Versuch 2 - Statistik}
% \newcommand{\Fachbereich}{Praktikum Messtechnik}
% \input{../packages/tu_header}

\newcommand{\institut}{Institut f\"ur Telekommunikationssysteme}
\newcommand{\fachgebiet}{Nachrichten\"ubertragung}
\newcommand{\veranstaltung}{Praktikum Nachrichten\"ubertragung}
\newcommand{\pdfautor}{\"Ozg\"u Dogan (326 048), Boris Henckell (325 779)}
\newcommand{\autor}{\"Ozg\"u Dogan (326 048)\\ Boris Henckell (325 779)}
\newcommand{\gruppe}{Gruppe: D 03}
%\newcommand{\betreuer}{Betreuer: Mahmoud Felk}


\newcommand{\pdftitle}{Nachrichten\"ubertragung\ Praktikum\ 01}
\newcommand{\prototitle}{Praktikum 01 \\ Einf\"uhrung in MATLAB}


\input{../../packages/tu_header_8}

%---------------------------------------------------------------------
%---------------------------------------------------------------------
%---------------------------------------------------------------------
\section{Einleitung}

Dieses Praktikum diente hauptsächlich dem Einarbeiten in die Laborumgebung und
dem Umgang der verwendeten Messger\"ate. Außerdem wurde der Einsatz von MatLab
geübt und einige wichtige Signale implementiert und untersucht.


\section{Vorbereitungsaufgaben}

    \begin{quote}
   	
        \subsection{Das Specktrum eines unendlich ausgedehnten Rechteckkamms}
        \begin{quote}
            Das besagt Rechteckkamm setzt sich wie folgt zusammen:
          
            \vspace{1em}
            
            \begin{equation*}
            	\begin{split}
            		u(t) = A \cdot &\sqcap_{\alpha T}(t) \ast \delta_T(t)\ , \\
                    0 < &\alpha < 1
            	\end{split}
            \end{equation*}
          	
          	Die Faltung im Zeitbereich, welche durch den Stern dargestellt wird,
          	entspricht einer Multiplikation der fouriertransformierten Faktoren
          	im Frequenzbereich. Daher gilt:
          	
          	\vspace{1em}
          	
          	\begin{equation*}
            	\begin{split}
            		F\{ A \cdot \sqcap_{\alpha T}(t) \} &= A \alpha T \cdot si(\frac{\omega \alpha T}{2})\\
                    F\{ \delta_T(t) \} &= \omega_T \delta_{\omega_T} (\omega)\\
                    &= \frac{2 \pi}{T} \delta_{\omega_T} (\omega)
            	\end{split}
            \end{equation*}
          	          
          	Daraus folgt:
          	
          	\vspace{1em}
          	
          	\begin{equation*}
            	\begin{split}
            		\Rightarrow
            		F\{ A \cdot &\sqcap_{\alpha T}(t) \ast \delta_T(t)\} = A \alpha T \cdot si(\frac{\omega \alpha T}{2})
            		\cdot \omega_T \delta_{\omega_T} (\omega)
            	\end{split}
            \end{equation*}
          	
   		 \end{quote}  	
	\end{quote}
         	
%--------------------------------------------------------------------
%--------------------------------------------------------------------            
\section{Praktische Aufgaben}

    \begin{quote}
        \subsection{Übungen für den Umgang mit MatLab}
		\begin{quote}
    		\subsubsection{Rechteck}
            \begin{quote}
                Um schnellstmöglich den Umgang mit MatLab zu erlernen, wurde uns eine Datei
                names rechteck.m und Aufgabe1.m vorgegeben. Damit konnten wir einen
                Rechtecksignal der Spannung mit einer Amplitude von 2V und einer
                vorgegebenen Länge simulieren.\\
                Dieses Signal wird anschließend Fourier transformiert. Dannach werden das Signal im Zeitbereich sowie der
                Amplitudengang und der Phasengang des Fourier transformierten Signals geplottet 
                
                \vspace{1em}
                
                Als nächstes wurde der Frequenzbereich für die
                Spektren im Code von Aufgabe1.m so verändert, dass wir den Bereich zwischen $0$
                und $4kHz$ darstellen konnten. Dazu wurden die Achsenwerte für die x-Achse
                (Frequenzbereich) und für die y-Achse (Amplituden) in den plot Befehlen
                überarbeitet.\\
                Außerdem haben wir untersucht, welche Auswirkunden die Veränderung des
                Tastgrads $\alpha$ hat. Die Spektren und Amplituden wurden für die Werte der
                folgenden Tabelle simuliert und untersucht.
                
                \vspace{1em}
                
                    \begin{center}
                         \begin{tabular}{|c|c|c|c|}
                             
                          \hline
                           $\alpha $ & $f[Hz]$ & Amplitude $[V]$ & Signaldauer
                           $T_{ges}[s]$\\ \hline
                           0.1 & 50 & 2 & 0.5 \\ \hline
                           0.5 & 50 & 2 & 0.5 \\ \hline
                           0.9 & 50 & 2 & 0.5 \\ \hline           
           
                         \end{tabular}
                     \end{center}
                  \vspace{1em}
                
                Als nächstes wurde die Amplitude so umgestellt, dass sie in dB angegeben
                wurde. Dies realisierten wir mit der ``log'' Funktion in MatLab, die die
                Umrechnung des Betragsspektrums ermöglicht.\\
                Die Funktion lautete dementsprechend:\\
                
                \begin{equation*}
                    \begin{split}
                        y_{DFT}_{abs} = 10*LOG10(abs(y_DFT)/N);     
                    \end{split}
                \end{equation*}
        
                Mit der ``ylim'' Funktion konnten wir dann die y-Achse zwischen $-30 dB$
                und $5 dB$ limitieren. Dafür verwendeten wir das Kommando:\\
                
                \begin{center}
                
                \begin{align*}
                    ylim ([-30 \ 5]);
                \end{align*}
                
                \end{center}
                
                Der komplette Code rechteck.m und Aufgabe1.m ist im Anhang zu finden.\\
                
            \end{quote}
            
            \subsubsection{Dreieck}
            \begin{quote}
                Nachdem das Rechecksignal abgeschlossen wurde, schreiben wir diesen Code zu
                einer dreieck.m Datei um. Das Ziel dabei war es die Funktion so zu
                modifizieren, dass eine bipolare Dreickfolge erstellt wurde. Der Taskgrad
                $\alpha$ sollte wieder variabel mit den Werten $\alpha = 0.1$, $\alpha = 0.5$
                und $\alpha = 0.9$ sein. Dies gelang uns, indem wir die if-Bedingungen in der
                rechteck.m Datei wie folgt umschrieben:
    
\begin{lstlisting}
if(t_{rel} < alpha*T_{periode})
res(k) = -A+((2*A)*t_{rel}/(alpha*T_{periode}));
else
res(k) = A+((alpha*T_{periode}*2*A)/((1-alpha)*T_{periode}))
         -((2*A)*t_{rel}/((1-alpha)*T_{periode}));
\end{lstlisting}  
                
                \vspace{1em}
                
                Die erste Bedingung beschreibt die aufsteigende Gerade solange die
                Laufvariable kleiner ist als $\alpha \cdot T_{periode}$. Falls diese Bedingung
                nicht erfüllt ist, das heißt, die Laufvariable ist größer, wird die Kurve mit
                der abfallenden Gerade fortgesetzt. Wichtig bei der Modifizierung war es, die Steigungen 
                und die y-Achsenabschnitte der Getraden korrekt zu bestimmen, damit das
                simulierte Ergebnis nicht verzerrt oder falsch wurde.
                
                Wir führten die Simulation mit den Tastgradwerten $\alpha = 0.3$, $\alpha = 0.5$
                und $\alpha = 0.9$ durch.
                
                Der vollständige Code zu dreieck.m ist im Anhang zu finden.
                
            \end{quote}
            
            \subsubsection{Cosinus}
            \begin{quote}
                Nach dem Dreiecksignal wurde die rechteck.m Datei nochmal so modifiziert, dass
                es eine bipolare Cosinusfolge erstellte.
                Dazu wurde der Zeitbereich in zwei Teile aufgeteilt, einmal in $t_{rel} <
                \alpha T_p$ und einmal in $t_{rel} > \alpha T_p$. Die Folge wurde mit den
                folgenden Funktionen beschrieben:\\
            
\begin{lstlisting}
if(t_rel < alpha*T_periode)
   res(k) = (A*cos((pi*t_rel)/(alpha*T_periode)));
else
   res(k) = (-A*cos((pi*(t_rel-alpha*T_periode))/((1-alpha)*T_periode)));
\end{lstlisting} 
            
                Erneut wurde das Ergebnis mit drei unterschiedlichen Tastgraden simuliert.\\
                
                Auch der Code zu cosinus.m ist im Anhang zu finden.\\
                
            \end{quote}
        \end{quote}
        
    \subsection{Übungen für den Umgang mit USB-Oszilloskop und Funktionsgenerator}  
    \begin{quote}
    
    Die Bearbeitung der kommenden Aufgaben diente dazu, um einen Einblick in die
    grundlegenden Funktionen und die Funktionsweise des USB-Oszilloskops und des
    Funktionsgenerators. Für den Aufbau reichte es den Ausgang des
    Funktionsgenerator mit hilfe eines Koaxialkabels mit dem A-Kanal-Eingang des Oszillioskops 
    zu verbinden.\\
    
    Als erstes wurde dann am Funktionsgenerator eine $100 Hz$
    Sinusschwingung mit einer Amplitude von grob $2V$ eingestellt. Die Darstellung am
    Oszilloskop wurde zunächst mit dem ``Automatische-Einrichtung''-Button im
    Oszilloskopfenster skaliert. Danach konnten wir mit dem entsprechenden Button
    die steigende Flanke des Eingangssignals triggern und die Zeit- und
    Amplitudenauflösung des Oszilloskops verändern um mit den Einstellungen des
    PicoScopes vertrauter zu werden.\\
    
    Mit dem Funktionsgenerator wurde jeweils ein Rechteck- ein Dreieck- und ein Sinussignal erzeugt, mit dem
    USB-Oszilloskop aufgenommen und anschließend als .mat abgespeichert. Diese Ergebnisse wurden anschließend in Mathlab
    Fouiertransformiert und geplottet. Im nächsten Abschnitt werden diese Ergebnisse mit den Simulierten Ergebnissen
    verglichen.\\
    
    Außerdem wurde von allen neun Signalen der RMS gemessen. 
    
    
    \end{quote}     
\begin{quote}       
\end{quote} 


\end{quote}
    
        
        
	
\subsection{Ergebnisse}
\begin{quote}
    \subsection{Rechteckesignal}
    \begin{quote}
        Zuerst vergleichen wir das Rechtecksignal. Wie in der Vorbereitungsaufgabe beschieben erwarten wir bei der
        Fouriertransformierten eine Si-Funktion im Amplitudenfrequenzgang.


                        %4 Grafiken:
            \begin{center}
            \begin{tabular}{ll}

            \hspace{-12em}
                \begin{minipage}{0.6\textwidth}

                    \begin{figure}[H]
                        \label{fig:}
                        \includegraphics[scale=0.25]{./Bilder/recht_alpha1.png} %FIXME [width=640px,
                         %height=474px]
                        \caption{Rechtecksignalspektrum für $\aplpha = 0,1$ - Simuliert}
                    \end{figure}

                \end{minipage}
                \begin{minipage}{0.6\textwidth}

                    \begin{figure}[H]
                        \label{fig:}
                        \includegraphics[scale=0.3]{./Bilder/recht_alpha1_-_gemessen.png} %FIXME [width=640px,
                         %height=474px]
                        \caption{Rechtecksignalspektrum für $\aplpha = 0,1$ - gemessen}
                    \end{figure}
                \vspace{-1.5em}

                \end{minipage}

            \end{tabular}
            \end{center}

                        %4 Grafiken:
            \begin{center}
            \begin{tabular}{ll}

            \hspace{-12em}
                \begin{minipage}{0.6\textwidth}

                    \begin{figure}[H]
                        \label{fig:}
                        \includegraphics[scale=0.25]{./Bilder/recht_alpha5.png} %FIXME [width=640px, height=474px]
                        \caption{Rechtecksignalspektrum für $\aplpha = 0,5$ - Simuliert}
                    \end{figure}

                \end{minipage}
                \begin{minipage}{0.6\textwidth}

                     \begin{figure}[H]
                        \label{fig:}
                        \includegraphics[scale=0.3]{./Bilder/recht_alpha5_-_gemessen.png} %FIXME [width=640px,
                        % height=474px]
                        \caption{Rechtecksignalspektrum für $\aplpha = 0,5$ - gemessen}
                    \end{figure}
               \vspace{-1.5em}

                \end{minipage}

            \end{tabular}
            \end{center}

                        %4 Grafiken:
            \begin{center}
            \begin{tabular}{ll}

            \hspace{-12em}
                \begin{minipage}{0.6\textwidth}

                    \begin{figure}[H]
                        \label{fig:}
                        \includegraphics[scale=0.25]{./Bilder/recht_alpha9.png} %FIXME [width=640px, height=474px]
                        \caption{Rechtecksignalspektrum für $\aplpha = 0,9$ - Simuliert}
                    \end{figure}

                \end{minipage}
                \begin{minipage}{0.6\textwidth}

                   \begin{figure}[H]
                        \label{fig:}
                        \includegraphics[scale=0.3]{./Bilder/recht_alpha9_-_gemessen.png} %FIXME [width=640px,
                        % height=474px]
                        \caption{Rechtecksignalspektrum für $\aplpha = 0,9$ - gemessen}
                    \end{figure}
                 \vspace{-1.5em}

                \end{minipage}

            \end{tabular}
            \end{center}

                        
            Die erwartete Si-Funktion im Amplitudenfrequenzgang lässt sich sowohl in den drei Simulierten Graphen als
            auch im den gemessenen Graphen erkennen. Da die simulierten Signale
            bis zu Frequenzen von $4 kHz$ geplottet wurden, die gemessenen
            jedoch nur um die $100 Hz$ lassen sich die 6 Graphen nicht direkt übereinanderlegen.\\
            Es lässt sich jedoch erkennen, dass die Si-Funktion für steigendes $\alpha$ immer schmaler wird. Der
            zusammenhang liegt daran, dass das Zeitsignal mit steigendem $\alpha$ immer breiter wird. Und wenn ein
            Signal im Zeitbereich breiter wird, wird es gleichzeitig im Frequenzbereich schmaler.\\
            
            Da die Leistung sich proportional zu der Fläche des Signals verhält steigt bei dem Rechtecksignal die
            Leistung mit steigendem $\alpha$.
                     
            \begin{center}
                  \begin{tabular}{|c|c|}
                  \hline
                   $\alpha $ &  Mittelwert [mV] \\ \hline 
                   0.1 &  529 \\ \hline
                   0.5 &  833 \\ \hline
                   0.9 &  955 \\ \hline           
                 \end{tabular}
                       \caption{RMS des Rechtecksignals}
                        \label{tablelabel1}
            
            \end{center}
        
    \end{quote}
    
    \subsection{Dreiecksignal}
    \begin{quote}
    
    	Nun betrachten wir die Ergebnisse des Dreiecksignals.

                        %4 Grafiken:
            \begin{center}
            \begin{tabular}{ll}

            \hspace{-12em}
                \begin{minipage}{0.6\textwidth}

                    \begin{figure}[H]
                        \label{fig:}            
                        \includegraphics[scale=0.25]{./Bilder/drei_alpha1.png} %FIXME [width=640px,
                        % height=474px]
                        \caption{Dreiecksignalspektrum für $\aplpha = 0,1$ - Simuliert}
                    \end{figure}

                \end{minipage}
                \begin{minipage}{0.6\textwidth}

                    \begin{figure}[H]
                        \label{fig:}            
                        \includegraphics[scale=0.3]{./Bilder/drei_alpha1_-_gemessen.png} %FIXME [width=640px,
                        % height=474px]
                        \caption{Dreiecksignalspektrum für $\aplpha = 0,1$ - gemessen}
                    \end{figure}                
                \vspace{-1.5em}

                \end{minipage}

            \end{tabular}
            \end{center}

                        %4 Grafiken:
            \begin{center}
            \begin{tabular}{ll}

            \hspace{-12em}
                \begin{minipage}{0.6\textwidth}

                    \begin{figure}[H]
                        \label{fig:}            
                        \includegraphics[scale=0.25]{./Bilder/drei_alpha5.png} %FIXME [width=640px, height=474px]
                        \caption{Dreiecksignalspektrum für $\aplpha = 0,5$ - Simuliert}
                    \end{figure}

                \end{minipage}
                \begin{minipage}{0.6\textwidth}

                    \begin{figure}[H]
                        \label{fig:}            
                        \includegraphics[scale=0.3]{./Bilder/drei_alpha5_-_gemessen.png} %FIXME [width=640px,
                        % height=474px]
                        \caption{Dreiecksignalspektrum für $\aplpha = 0,5$ - gemessen}
                    \end{figure}                
               \vspace{-1.5em}

                \end{minipage}

            \end{tabular}
            \end{center}

                        %4 Grafiken:
            \begin{center}
            \begin{tabular}{ll}

            \hspace{-12em}
                \begin{minipage}{0.6\textwidth}

                    \begin{figure}[H]
                        \label{fig:}            
                        \includegraphics[scale=0.25]{./Bilder/drei_alpha9.png} %FIXME [width=640px, height=474px]
                        \caption{Dreiecksignalspektrum für $\aplpha = 0,9$ - Simuliert}
                    \end{figure}

                \end{minipage}
                \begin{minipage}{0.6\textwidth}

                    \begin{figure}[H]
                        \label{fig:}            
                        \includegraphics[scale=0.3]{./Bilder/drei_alpha9_-_gemessen.png} %FIXME [width=640px,
                        % height=474px]
                        \caption{Dreiecksignalspektrum für $\aplpha = 0,9$ - gemessen}
                    \end{figure}                
                 \vspace{-1.5em}

                \end{minipage}

            \end{tabular}
            \end{center}

            
            Ein Dreiecksignal lässt sich im Zeitbereich als eine Faltung von
            zwei Rechtecksignalen darstellen. Überträgt man diese Faltung in den
            Frequenzbereich, wird daraus eine Multiplikation zweier
            si-Funktionen, wodurch ein Ergebnis von einer $si^2(x)$ Funktion
            ensteht. Diese Erwartung wird im Amplitudenspektrum auch erfüllt.
            Man sieht nämlich den Verlauf einer quadrierten si-Funktion ohne
            negativen Anteil und mit wachsender Freuquenz gegen null strebender
            Amplitude.\\
            
            Das $\alpha$ bewirkt lediglich nur eine Verschiebung des höchsten
            Punktes im Zeitbereich. Dadurch verändert sich die Fläche unter dem
            Dreicksignal nicht, wodurch auch die Leistung des Signals nicht
            steigt. Dies kann man auch an den unveränderten RMS-Messwerten
            sehen.
           
                     
            \begin{center}
                  \begin{tabular}{|c|c|}
                  \hline
                   $\alpha $ &  Mittelwert [mV] \\ \hline 
                   0.1 &  567 \\ \hline
                   0.5 &  566 \\ \hline
                   0.9 &  567 \\ \hline           
                 \end{tabular}
                       \caption{RMS des Rechtecksignals}
                        \label{tablelabel1}
             \end{center}
        
    \end{quote}
    
    \subsection{Cosinussignal}
    \begin{quote}
    
    	Zuletzt wird das Cosinussignal mit den drei Tastgraden simuliert und
    	durchgemessen.
        
        
                        %4 Grafiken:
            \begin{center}
            \begin{tabular}{ll}

            \hspace{-12em}
                \begin{minipage}{0.6\textwidth}

                    \begin{figure}[H]
                        \label{fig:}            
                        \includegraphics[scale=0.25]{./Bilder/cos_alpha1.png} %FIXME [width=640px,
                        % height=474px]
                        \caption{Cosinussignalspektrum für $\aplpha = 0,1$ - Simuliert}
                    \end{figure}

                \end{minipage}
                \begin{minipage}{0.6\textwidth}

                    \begin{figure}[H]
                        \label{fig:}            
                        \includegraphics[scale=0.3]{./Bilder/cos_alpha1_-_gemessen.png} %FIXME [width=640px,
                        % height=474px]
                        \caption{Cosinussignalspektrum für $\aplpha = 0,1$ - gemessen}
                    \end{figure}                
                \vspace{-1.5em}

                \end{minipage}

            \end{tabular}
            \end{center}

                        %4 Grafiken:
            \begin{center}
            \begin{tabular}{ll}

            \hspace{-12em}
                \begin{minipage}{0.6\textwidth}

                    \begin{figure}[H]
                        \label{fig:}            
                        \includegraphics[scale=0.25]{./Bilder/cos_alpha5.png} %FIXME [width=640px, height=474px]
                        \caption{Cosinussignalspektrum für $\aplpha = 0,5$ - Simuliert}
                    \end{figure}

                \end{minipage}
                \begin{minipage}{0.6\textwidth}

                    \begin{figure}[H]
                        \label{fig:}            
                        \includegraphics[scale=0.3]{./Bilder/cos_alpha5_-_gemessen.png} %FIXME [width=640px,
                        % height=474px]
                        \caption{Cosinussignalspektrum für $\aplpha = 0,5$ - gemessen}
                    \end{figure}                
               \vspace{-1.5em}

                \end{minipage}

            \end{tabular}
            \end{center}

                        %4 Grafiken:
            \begin{center}
            \begin{tabular}{ll}

            \hspace{-12em}
                \begin{minipage}{0.6\textwidth}

                    \begin{figure}[H]
                        \label{fig:}            
                        \includegraphics[scale=0.25]{./Bilder/cos_alpha9.png} %FIXME [width=640px, height=474px]
                        \caption{Cosinssignalspektrum für $\aplpha = 0,9$ - Simuliert}
                    \end{figure}

                \end{minipage}
                \begin{minipage}{0.6\textwidth}

                    \begin{figure}[H]
                        \label{fig:}            
                        \includegraphics[scale=0.3]{./Bilder/cos_alpha9_-_gemessen.png} %FIXME [width=640px,
                        % height=474px]
                        \caption{Cosinussignalspektrum für $\aplpha = 0,9$ - gemessen}
                    \end{figure}                
                 \vspace{-1.5em}

                \end{minipage}

            \end{tabular}
            \end{center}
        
            
            Bei einem $\alpha$ von $0.5$ ist der höchste Punkt des Signals genau
            in der Mitte der Amplitude. Daher ensteht in der Simulation auch
            auch ein unverzerrter Sinus im Zeitbereich und ein einzelner
            Deltaimpuls im Frequenzbereich. Dieser Deltaimpuls ist auch in
            der realen Messung relativ deutlich zu sehen, wodurch die
            Erwartungen der Simulation in der Messung erfüllt werden.\\
            
            Bei einem $\alpha$ von $0.1$ ist wiederum der höchste Punkt der Amplitude verschoben. 
            Das Sinussignal ist leicht verzerrt, weshalb die Fouriertransformierte 
            in der Simulation aus mehreren Deltaimpulsen besteht. Im
            Amplitudenspektrum der Messung sehen wir ebenfalls mehrere Deltaimpulse.\\
            
            Interessant ist hier, dass das Amplitudenspektrum der Simulation
            bei einem $\alpha$ von $0.9$ genau dem Amplitudenspektrum des
            Sinussignals bei $\alpha = 0.1$ entspricht. Auch in der Messung
            liefert die Fouriertransformation ein sehr ähnliches Ergebnis. Der
            Grund dafür liegt in der Differenz des $\alpha$ zum Wert $0.5$
            (beide Male $0.4$).
            Das Signal wird in beiden Fällen zwar in unterschiedliche Richtungen,
            dafür aber gleich stark verzerrt. Daher ändert sich die
            Fouriertransformation nicht und das Amplitudenspektrum bleibt in
            beiden Simulationen sowie Messungen relativ ähnlich.
                                         
            \begin{center}
                  \begin{tabular}{|c|c|}
                  \hline
                   $\alpha $ &  Mittelwert [mV] \\ \hline 
                   0.1 &  700 \\ \hline
                   0.5 &  705 \\ \hline
                   0.9 &  706 \\ \hline           
                 \end{tabular}
                       \caption{RMS des Cosinussignals}
                        \label{tablelabel1}
                        
            \end{center}
        
    \end{quote}


    
    


    
        
\end{quote}


%--------------------------------------------------------------------
%--------------------------------------------------------------------
\section{Matlab-Code}
\begin{quote}
    \subsection{Aufgabe1.m bearbeitet}
    \begin{quote}
            \lstinputlisting[
            caption={Aufgabe1 - Matlab-script},
            label=lst:Matlab]
            {./Matlab/Aufgabe1.m}
    \end{quote}
    \subsection{rechteck.m bearbeitet}
    \begin{quote}
            \lstinputlisting[
            caption={Rechteck - Matlab-script},
            label=lst:Matlab]
            {./Matlab/rechteck.m}
    \end{quote}
    \subsection{dreieck.m bearbeitet}
    \begin{quote}
            \lstinputlisting[
            caption={Dreieck - Matlab-script},
            label=lst:Matlab]
            {./Matlab/dreieck.m}        
    \end{quote}
    \subsection{cosinus.m bearbeitet}
    \begin{quote}
            \lstinputlisting[
            caption={Cosinus - Matlab-script},
            label=lst:Matlab]
            {./Matlab/cosinus.m}        
    \end{quote}                 	
\end{quote}


\end{document}